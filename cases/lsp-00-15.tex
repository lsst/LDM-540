\subsection{LSP-00-15: Execution of basic catalog queries in the Portal}
\label{lsp-00-15}

\subsubsection{Requirements}

???

\subsubsection{Test items}

This test will test the functional requirements to be able to perform a range of basic queries through the Portal Aspect of the LSP:

\begin{itemize}

  \item{Cone searches on the Object-like, ForcedSource-like, and Source-like WISE tables;}
  \item{Multi-target cone searches;}
  \item{Form-based searches for exact equality, e.g., for row IDs; and}
  \item{Form-based searches for sets of object attributes.}

\end{itemize}

In addition, it tests the ability to download tabular query results from the Portal Aspect.

\subsubsection{Intercase dependencies}

As the Portal's implementation of queries of this nature is based on the invocation of the same APIs that will be invoked by LSP-00-05 and LSP-00-10, successful completion of this test case is likely contingent on the success of those two - but does not explicitly depend on them.

Some of the execution of these test cases relies on features of the UI which are formally tested in case LSP-00-20.

\subsubsection{Environmental needs}

The test must, and can only, be carried out on the PDAC instance of the Science Platform.
It relies on the availability of the IRSA deployment of the WISE and ALLWISE data to use in order to perform comparisons on content and performance.

The test may be performed from any client computer on the public Internet.
The computer used, and the network connection involved, shall be documented in the test report.

The test requires access via the NCSA VPN from the client to the PDAC data services 
(which means that the performance of the VPN is a contributor to the performance observed).

The client computer must have an up-to-date mainstream Web browser, the identity of which shall be documented in the test report.

\subsubsection{Input specification}

The test requires the presence of the following WISE tables on the PDAC systems:

\begin{itemize}

  \item{The Source Catalog table from the AllWISE data release;}
  \item{The Multi-Epoch Photometry (MEP) table from the AllWISE data release;}
  \item{The Single-exposure Source Database from the WISE All-Sky data release (i.e., from the full-cryo 4-band phase of the original mission);}
  \item{The Single-exposure Source Database from the NEOWISE Reactivation 2015 data release (i.e., from the first year of the 2-band post-reactivation phase of the mission).}
\end{itemize}


\subsubsection{Output specification}

The output will consist of:

\begin{itemize}
  \item{A written log of the tests performed on the GUI-based Portal Aspect; and}
  \item{A collection of the files that resulted from data download actions performed as part of the tests.}
\end{itemize}


\subsubsection{Procedure}

\begin{enumerate}

  \item{Log host and networking details of the client host to be used.
 Log the Web browser version to be used.}
  \item{Establish VPN connectivity to the PDAC at NCSA.}
  \item{Manually perform, and log, the following steps against the Portal Aspect:
    \begin{enumerate}
      \item{Navigate to the PDAC Portal.  Log the URL used to do so.}
      \item{Perform a cone search around (ra=0,dec=0), radius 300 arcseconds, in each of the Object-like, ForcedSource-like, and a Source-like catalog.
 Choose a row from each search and record the primary key value for each for later use.
 Take screen shots of the search form and of the results of the three searches.
 Record the wall clock time required for the searches, if long enough to measure.}
      \item{Perform a multi-object cone search based on the coordinates in the file LDM-540/test\textunderscore scripts/lsp-00-15.coords in the Object-like table.
 Take screen shots of the search form and of the results of the search.
 Record the wall clock time required for the searches, if long enough to measure.}
      \item{Perform a search on each of the Object-like, ForcedSource-like, and Source-like catalogs for the IDs previously saved.
 Confirm that each search is successful and returns the same information as in the original search from which the ID was taken.
 Perform a search on the ForcedSource-like catalog using the ID from the Object-like catalog.
 Confirm that a time series of measurements of that object in multiple epochs is returned.
 Take screen shots of the search forms and of the results of the searches.
 Record the wall clock time required for the searches, if long enough to measure.}
      \item{On each of the Object-like catalog and a Source-like catalog, by performing searches over small regions of sky and exploring the results, choose a set of attributes and search parameters which should select a relatively small number of rows ($<100,000$) when applied to the entire sky.
 This may require some iterative experimentation at increasingly larger scales.
 Take screen shots of the final search forms and of the results of the searches.
 Record the wall clock time required for the searches, if long enough to measure.}
    \end{enumerate}
  }

\end{enumerate}
