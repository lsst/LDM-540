\subsection{LSP-00-05: Demonstration of low-volume and/or indexed queries against the WISE data via API}
\label{lsp-00-05}

\subsubsection{Requirements}

Requirement: DMS-LSP-REQ-004, [existence of the] API Aspect

\subsubsection{Test items}

This test will check that the following low-volume queries can be performed against the WISE catalogs via the API Aspect.

\begin{itemize}

  \item{Small cone searches against the Object-like, ForcedSource-like, and Source-like tables; and}
  \item{Searches by exact ID matching against the Object-like, ForcedSource-like, and Source-like tables.}

\end{itemize}

The tests will record their performance for comparison against similar queries in the production WISE archive at IRSA,
and the returned data will be compared to that for similar queries against the API services provided by IRSA.

\subsubsection{Intercase dependencies}

Several of the queries performed under LSP-00-15, from the Portal Aspect, cover the same ground and will depend on their successful completion under this test case.

\subsubsection{Environmental needs}

The test must, and can only, be carried out on the PDAC instance of the Science Platform.
It relies on the availability of the IRSA deployment of the WISE and ALLWISE data to use in order to perform comparisons on content and performance.

The test may be performed from any client computer on the public Internet.
The computer used, and the network connection involved, shall be documented in the test report.

The test requires access via the NCSA VPN from the client to the PDAC data services 
(which means that the performance of the VPN is a contributor to the performance observed).

The client computer must have an up-to-date mainstream Web browser, the identity of which shall be documented in the test report.

The tests will be captured in a Jupyter notebook run locally on the client computer, so the client system must have a recent production version of Jupyter installed.
(The test does \emph{not} depend on features of JupyterLab in any way, only the standard Notebook.)


\subsubsection{Input specification}

The test requires, the presence of the following WISE tables on the PDAC systems:

\begin{itemize}

  \item{The Source Catalog table from the AllWISE data release;}
  \item{The Multi-Epoch Photometry (MEP) table from the AllWISE data release;}
  \item{The Single-exposure Source Database from the WISE All-Sky data release (i.e., from the full-cryo 4-band phase of the original mission); and}
  \item{The Single-exposure Source Database from the NEOWISE Reactivation 2015 data release (i.e., from the first year of the 2-band post-reactivation phase of the mission).}
\end{itemize}

(The 3-band and 2-band Single-exposure Source Database tables from the pre-hibernation mission are not queried in this test case, for the sake of simplicity.
These tables are smaller than the two Single-exposure Source Database tables that are included in the test case.)

\subsubsection{Output specification}

The output will consist of:

\begin{itemize}
  \item{A Jupyter notebook of the API Aspect tests performed; and}
  \item{A collection of the files that resulted from data download actions performed as part of the tests.}
\end{itemize}


\subsubsection{Procedure}

\begin{enumerate}

  \item{Clone the Github lsst/LDM-540 package.  Record the SHA1 for the version of the package to be used.  [Additional procedures, e.g., tagging, are still to be confirmed.]}
  \item{Log host and networking details of the client host to be used.
 Log the Web browser version to be used.
 Log the version of Jupyter to be used.}
  \item{Establish VPN connectivity to the PDAC at NCSA.}
  \item{Execute the LDM-540/test\textunderscore scripts/lsp-00-05.ipynb notebook to perform the tests listed above against the API Aspect.}
  \item{Preserve any outputs of the script that are in the form of files outside the notebook.}

\end{enumerate}
