\subsection{LSP-00-25: Image metadata, image, and image cutout queries}
\label{lsp-00-25}

\subsubsection{Requirements}

???

\subsubsection{Test items}

This test will check basic functionality related to image search and retrieval, via both the API Aspect and the Portal Aspect of the LSST Science Platform:

\begin{itemize}

  \item{Searching for images containing a specified point;}
  \item{Displaying selected images;}
  \item{Obtaining and displaying image cutouts at a specified point; and}
  \item{Downloading selected images and image cutouts.}

\end{itemize}

Because of limited staff resources, these tests will be based on the original PDAC dataset, the LSST Summer 2013 processing of the SDSS Stripe 82 data.
The image data for the WISE and NEOWISE missions have not been loaded into PDAC.

\subsubsection{Intercase dependencies}

LSP-00-30 relies on functionality explored by this test case.

\subsubsection{Environmental needs}

The test must, and can only, be carried out on the PDAC instance of the Science Platform.

The test may be performed from any client computer on the public Internet.
The computer used, and the network connection involved, shall be documented in the test report.

The test requires access via the NCSA VPN from the client to the PDAC data services 
(which means that the performance of the VPN is a contributor to the performance observed).

The client computer must have an up-to-date mainstream Web browser, the identity of which shall be documented in the test report.

The API Aspect tests will be captured in a Jupyter notebook run locally on the client computer, so the client system must have a recent production version of Jupyter installed.
(The test does \emph{not} depend on features of JupyterLab in any way, only the standard Notebook.)


\subsubsection{Input specification}

The test requires the presence of the following SDSS data, from the Summer 2013 LSST processing, in the PDAC systems:

\begin{itemize}

  \item{The catalog of coadded image tiles;}
  \item{The catalog of calibrated single-epoch images, including photometric zero points;}
  \item{The coadded image data; and}
  \item{The single-epoch calibrated image data.}
\end{itemize}

Note that a significant fraction of the single-epoch calibrated images are known to have been lost between 2013 and the loading of the data into PDAC.
The test report will note where that affects the result.

\subsubsection{Output specification}

The output will consist of:

\begin{itemize}
  \item{A Jupyter notebook of the API Aspect tests performed;}
  \item{A written log of the tests performed on the GUI-based Portal Aspect; and}
  \item{A collection of the files that resulted from data download actions performed as part of the tests.}
\end{itemize}


\subsubsection{Procedure}

\begin{enumerate}

  \item{Clone the Github lsst/LDM-540 package.  Record the SHA1 for the version of the package to be used.  [Additional procedures, e.g., tagging, are still to be confirmed.]}
  \item{Log host and networking details of the client host to be used.
 Log the Web browser version to be used.
 Log the version of Jupyter to be used.}
  \item{Establish VPN connectivity to the PDAC at NCSA.}
  \item{Execute the LDM-540/test\textunderscore scripts/lsp-00-25.ipynb notebook to perform the tests listed above against the API Aspect.}
  \item{Preserve any outputs of the script that are in the form of files outside the notebook.}
  \item{Manually perform, and log, the following steps against the Portal Aspect:
    \begin{enumerate}
      \item{Navigate to the PDAC Portal.  Log the URL used to do so.}
      \item{Perform image queries against the SDSS Stripe 82 coadded and single-epoch image data.
 These queries should match ones in the API Aspect test notebook.
 Document the query results with screen shots.}
      \item{Download an example of each type of image.}
      \item{Using the public SDSS archive, attempt to retrieve corresponding images and visually compare them.}
      \item{Perform image cutout requests for $300\times 300$ fields at the same targets used in the full-image queries.}
      \item{Download the cutouts.}
    \end{enumerate}
  }

\end{enumerate}
