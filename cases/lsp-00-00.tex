\subsection{LSP-00-00: Verification of the presence of the expected WISE data}
\label{lsp-00-00}

\subsubsection{Requirements}

The structure of the WISE data allows a test with tables analogous to Object, ForcedSource, and Source, 
so this test partially addresses the implied requirement that Qserv and the API and Portal Aspects be able to handle all the expected data types to be produced by LSST.

Requirement: DMS-LSP-REQ-0001, Access to All Released or Authorized Data Products; DMS-LSP-REQ-0005, Linkage of Aspects (limited to API-Portal linkage)

\subsubsection{Test items}

This test will check:

\begin{itemize}

  \item{That the expected tables are present in the database and accessible via the API Aspect and the Portal Aspect;}
  \item{That the tables are present with the expected schema as documented in the IPAC-provided WISE documentation;}
  \item{That the row counts in the tables are as expected;}
  \item{That the tables cover essentially the entire sky, as expected from the characteristics of the WISE mission.}

\end{itemize}

\subsubsection{Intercase dependencies}

All the further test cases rely on this one being met with near-complete success.

\subsubsection{Environmental needs}

The test must, and can only, be carried out on the PDAC instance of the Science Platform.
It relies on the availability of the IRSA deployment of the WISE and ALLWISE data to use in order to perform comparisons on content and performance.

The test may be performed from any client computer on the public Internet.
The computer used, and the network connection involved, shall be documented in the test report.

The test requires access via the NCSA VPN from the client to the PDAC data services 
(which means that the performance of the VPN is a contributor to the performance observed).

The client computer must have an up-to-date mainstream Web browser, the identity of which shall be documented in the test report.

The API Aspect tests will be captured in a Jupyter notebook run locally on the client computer, so the client system must have a recent production version of Jupyter installed.
(The test does \emph{not} depend on features of JupyterLab in any way, only the standard Notebook.)


\subsubsection{Input specification}

The test verifies, and therefore requires, the presence of the following WISE tables on the PDAC systems:

\begin{itemize}

  \item{The Source Catalog table from the AllWISE data release;}
  \item{The Multi-Epoch Photometry (MEP) table from the AllWISE data release;}
  \item{The Single-exposure Source Database from the WISE All-Sky data release (i.e., from the full-cryo 4-band phase of the original mission);}
  \item{The Single-exposure Source Database from the WISE 3-Band Cryo data release (i.e., from the 3-band, inner-cryogen-tank-only phase of the mission);}
  \item{The Single-exposure Source Database from the NEOWISE Post-Cryo data release (i.e., from the 2-band warm phase of the mission, pre-hibernation);and}
  \item{The Single-exposure Source Database from the NEOWISE Reactivation 2015 data release (i.e., from the first year of the 2-band post-reactivation phase of the mission).}
\end{itemize}


\subsubsection{Output specification}

The output will consist of:

\begin{itemize}
  \item{A Jupyter notebook of the API Aspect tests performed;}
  \item{A written log of the tests performed on the GUI-based Portal Aspect; and}
  \item{A collection of the files that resulted from data download actions performed as part of the tests.}
\end{itemize}


\subsubsection{Procedure}

\begin{enumerate}

  \item{Clone the Github lsst/LDM-540 package.  Record the SHA1 for the version of the package to be used.  [Additional procedures, e.g., tagging, are still to be confirmed.]}
  \item{Log host and networking details of the client host to be used.
 Log the Web browser version to be used.
 Log the version of Jupyter to be used.}
  \item{Establish VPN connectivity to the PDAC at NCSA.}
  \item{Execute the LDM-540/test\textunderscore scripts/lsp-00-00.ipynb notebook to perform the tests listed above against the API Aspect.}
  \item{Preserve any outputs of the script that are in the form of files outside the notebook.}
  \item{Manually perform, and log, the following steps against the Portal Aspect:
    \begin{enumerate}
      \item{Navigate to the PDAC Portal.  Log the URL used to do so.}
      \item{Navigate to the list of WISE catalogs available.  Preserve screen shots as needed to document that the UI offers access to the required catalogs.}
      \item{Perform a trivial query against each required catalog to confirm that the UI does indeed provide access.
 These queries should match ones in the API Aspect test notebook and will generally be small cone searches around (ra=0, dec=0).
 Document the query results with screen shots.
 It is not necessary at this time to exhaustively verify that the Portal provides access to every column of every table; spot checks should be performed and documented.
 (A later test will perform a full column-by-column verification, once test data more closely matching the final DPDD-driven LSST data products are available.)}
%     ...
    \end{enumerate}
  }

\end{enumerate}
