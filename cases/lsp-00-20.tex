\subsection{LSP-00-20: Operation of the UI for interaction with tabular data results}
\label{lsp-00-20}

\subsubsection{Requirements}

LSP-level Requirements: DMS-LSP-REQ-0014, Download Data; DMS-LSP-REQ-0017, Tabular Data Download File Formats

Portal-level Requirements: DMS-PRTL-REQ-0049, Display of Tabular Data; DMS-PRTL-REQ-0053, Row Selection of Tabular Data; partial coverage of DMS-PRTL-REQ-0050, Column Selection of Tabular Data; DMS-PRTL-REQ-0054, Paging of Tabular Data; DMS-PRTL-REQ-0055, XY Scatter Plots; partial coverage of DMS-PRTL-REQ-0056, Histograms

\subsubsection{Test items}

This test will test the functional requirements to be able to perform certain basic exploratory data analysis functions on tabular data results in the Portal Aspect UI:

\begin{itemize}

  \item{Sort tabular results;}
  \item{Filter tabular results based on the contents of columns;}
  \item{Perform per-row selections from a table;}
  \item{Display 1D histograms of selected attributes;}
  \item{Display 2D scatter plots of selected attributes;}
  \item{Perform graphical selections of rows from plots; and}
  \item{Download tabular query results reflecting sorting and selection.}

\end{itemize}

This test does not address the limits of scaling of these capabilities to large query results.
That will be addressed in future test specifications.
The test report should include notes on the sizes of results that were used.

\subsubsection{Intercase dependencies}

The ability to do these tests depends on the ability to perform tabular queries at all.

\subsubsection{Environmental needs}

The test must, and can only, be carried out on the PDAC instance of the Science Platform.
It relies on the availability of the IRSA deployment of the WISE and ALLWISE data to use in order to perform comparisons on content and performance.

The test may be performed from any client computer on the public Internet.
The computer used, and the network connection involved, shall be documented in the test report.

The test requires access via the NCSA VPN from the client to the PDAC data services 
(which means that the performance of the VPN is a contributor to the performance observed).

The client computer must have an up-to-date mainstream Web browser, the identity of which shall be documented in the test report.


\subsubsection{Input specification}

The test requires the presence of at least the Source Catalog table from the AllWISE data release.


\subsubsection{Output specification}

The output will consist of:

\begin{itemize}
  \item{A written log of the tests performed on the GUI-based Portal Aspect; and}
  \item{A collection of the files that resulted from data download actions performed as part of the tests.}
\end{itemize}


\subsubsection{Procedure}

\begin{enumerate}

  \item{Log host and networking details of the client host to be used.
 Log the Web browser version to be used.}
  \item{Establish VPN connectivity to the PDAC at NCSA.}
  \item{Manually perform, and log, the following steps against the Portal Aspect, recording the time required for a step if it is not trivial:
    \begin{enumerate}
      \item{Navigate to the PDAC Portal.  Log the URL used to do so.}
      \item{Perform a cone search around (ra=0,dec=0), of a radius that produces a result of at least $10,000$ records.
 (Note again that this is a \emph{functionality} test, not a scaling test, however.)}
      \item{Download the query result.
 Log the file type used for the download, and the file types offered.
 Confirm that the download includes the same number of rows reported by the UI.
 Confirm by sampling at the beginning and end that the sort order of the download and that shown in the UI are the same.}
      \item{Exercise sorting of the table, including both numeric and text columns.
 Download the results of performing a sort of each type.
 Confirm by sampling that the sort order of the download matches that in the UI.}
      \item{Verify using external tools (e.g., Unix command-line utilities) that the sort orders for text and numeric columns are appropriate.}
      \item{Exercise filtering of a tabular result based on values of attributes.
 Confirm that it is possible to download the filtered result.
 Verify using external tools that the filter was applied as expected, by comparing the filtered download with the original one.}
      \item{Perform selections of individual rows in the tabular result.
 Confirm that it is possible to download only the selected rows.}
      \item{Construct a 1D histogram of a numeric attribute.
 Using the functionality of the UI, sample a few bin values.
 Verify using external tools that the chosen bin values match those derived from the downloaded data.}
      \item{Construct a 2D scatter plot of selected numeric attributes.
 By inspecting extrema of those attributes in the table, and outlying points in the scatterplot, confirm that the plot appears to contain the expected data.
 (This is not meant to be an exhaustive test of completeness.)}
      \item{Perform a (rectangular) graphical selection from a 2D plot.
 Record the approximate dimensions of the selection as judged from the display.
 Record the exact values for the region boundaries reported by the UI.
 Download the selected data.
 Verify using external tools that the expected selection was applied.}
    \end{enumerate}
  }

\end{enumerate}
