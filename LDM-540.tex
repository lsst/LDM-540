\documentclass[DM,STS,toc]{lsstdoc}
\usepackage{enumitem}
\input meta.tex

\begin{document}

\def\product{LSST Science Platform}

\setDocCompact{true}

\title[Test Spec for \product]{\product~Test Specification}

\author{G. P. Dubois-Felsmann} 
\setDocRef{\lsstDocType-\lsstDocNum}
\setDocDate{2018-05-01}

\setDocAbstract {
This document describes the detailed test specification for the \product{}.
}

% Most recent last
\setDocChangeRecord{%
	\addtohist{1.0}{2018-05-01}{Adopted under \jira{RFC-468}.  Used to drive test LSP--00.}{G. P. Dubois-Felsmann}
	\addtohist{0.1}{2018-01-26}{Early drafting}{G. P. Dubois-Felsmann}
}

\setDocCurator{G.~P.~Dubois-Felsmann}
\setDocUpstreamLocation{\url{https://github.com/lsst/ldm-540}}
\setDocUpstreamVersion{\vcsrevision}

\maketitle

\section{Introduction}
\label{sec:intro}

This document specifies the test procedure for the \product{}.

The \product{} is the component of the LSST system which is responsible for providing data access and data analysis capabilities to users.

It is aimed at meeting the needs of several categories of users, including:

\begin{itemize}

  \item{Science users with LSST data rights;}
  \item{LSST Project, and later, Operations staff doing algorithm development and the associated validations;}
  \item{LSST Project staff engaged in Commissioning and related activities; and}
  \item{LSST Operations staff engaged in science validation and other data quality analyses}

\end{itemize}

A full description of this product is provided in \citeds{LDM-542}, with requirements enumerated in \citeds{LDM-554}.

\subsection{Objectives}
\label{sec:objectives}

This document builds on the description of LSST Data Management's approach to testing as described in \citeds{LDM-503} to describe the detailed tests that will be performed on the \product{} as part of the verification of the DM system.

It identifies test cases and procedures for the tests, and the pass/fail criteria for each test.

\subsection{Scope}
\label{sec:scope}

This document describes the test procedures for the following components of the LSST system (as described in \citeds{LDM-542}), and their deployment over the resources and services of the LSST Data Facility:

\begin{itemize}

  \item{The science database, especially its Qserv component;}
  \item{The API Aspect of the Science Platform, comprising:

    \begin{itemize}
      \item{Catalog query via TAP and related VO services;}
      \item{Image metadata query via TAP and SIAv2;}
      \item{Image retrieval and cutout generation;}
      \item{User Workspace database creation and access; and}
      \item{User Workspace file system access.}
    \end{itemize}
  }

  \item{The Portal Aspect of the Science Platform, comprising a set of Web-based tools for:

    \begin{itemize}
      \item{Data discovery for Project-generated and user-generated data;}
      \item{Catalog and image query;}
      \item{Image display;}
      \item{Catalog data visualization;}
      \item{Exploratory data analysis; and}
      \item{Alert subscription control.}
    \end{itemize}
  }

  \item{The Notebook Aspect of the Science Platform, comprising:

    \begin{itemize}
      \item{A deployment of the JupyterHub and JupyterLab interactive computing environments;}
      \item{Access to the API Aspect services from within that environment;}
      \item{Direct access to elements of the data systems underlying those services, e.g., access to the User File Workspace as a mounted filesystem rather than through the VOSpace API;}
      \item{A customizable, persistent user environment; and}
      \item{The provision of pre-built deployments of releases of the LSST Stack, usable to configure the computational environment provided by JupyterLab.}
    \end{itemize}

  }

\end{itemize}

\subsection{Applicable Documents}
\label{sec:docs}

\addtocounter{table}{-1}

\begin{tabular}[htb]{l l}
\citeds{LDM-148} & LSST DM System Architecture \\
\citeds{LDM-294} & LSST DM Organization \& Management \\
\citeds{LDM-503} & LSST DM Test Plan \\
\citeds{LDM-542} & LSST Science Platform Design \\
\citeds{LDM-554} & LSST Science Platform Requirements \\
\citeds{LSE-61}  & LSST DM Subsystem Requirements \\
\citeds{LSE-319} & LSST Science Platform Vision Document \\
\citeds{LSE-163} & LSST Data Products Definition Document \\
\end{tabular}

\subsection{References\label{sect:references}}
\renewcommand{\refname}{}
\bibliography{lsst,refs,books,refs_ads}

%\subsection{Definitions, acronyms, and abbreviations \label{sect:acronyms}} % include acronyms.tex generated by the acronyms.csh (GaiaTools)
%\input{acronyms}


%----------------------------------------------------
% TASK IDENTIFICATION - APPROACH
%----------------------------------------------------
\section{Approach}
\label{sec:approach}

The major activities to be performed are to:

\begin{itemize}

\item{Verify that the LSST Science Platform components are capable of performing the functions defined in the relevant DM System Requirements, \citeds{LSE-61}, and in the Science Platform Requirements, \citeds{LDM-554}.}
\item{Ensure that the components of the Science Platform match the documented design.}
\item{Test all the interfaces among components of the Science Platform.}
\item{Test all the interfaces between components of the Science Platform and other DM system components.}
\item{Within the limits of available integration and test hardware platforms and datasets, verify that the Science Platform components meet the performance requirements set forth in the above documents, or extrapolate appropriately from the test systems available to verify that the performance requirements should be met on a fully provisioned hardware platform.}
\item{Repeat these tests when the full hardware platform becomes available.}
\item{Ensure that the test procedures developed are also relevant to pre-deployment testing in the Operations era.}
\item{Ensure that the observed behavior of the Science Platform components when under test is consistent with the available documentation produced by their developers or by other authors.}

\end{itemize}

\subsection{Tasks and criteria}
\label{sec:tasks}

The following are the major items under test:

\begin{itemize}

\item{The LSST science database;}
\item{The API Aspect of the Science Platform, encompassing Web APIs for access to LSST Data Products, both within the science database and within the Data Backbone, and enabling the creation, sharing, and management of User Generated Data Products;}
\item{The Portal Aspect of the Science Platform, encompassing user interfaces for data discovery, retrieval, visualization, and exploratory data analysis, as well as an interface for the control of the alert subscription and ``mini-broker'' filtering mechanism; and}
\item{The Notebook Aspect of the Science Platform, providing interactive computing services for LSST science users and project-internal analysts.}

\end{itemize}

\subsection{Features to be tested}
\label{sec:feat2test}

\begin{itemize}

\item{Availability and, where relevant, proper interpretation of Prompt Data Products through each Aspect of the Science Platform;}
\item{Availability and, where relevant, proper interpretation of Data Release Data Products through each Aspect of the Science Platform;}
\item{Creation of, access to, and management of User Generated Data Products through each Aspect of the Science Platform;}
\item{Features related to authentication and authorization of users, including those related to custom access controls to User Generated Data Products;}
\item{Features related to the manageability of the Science Platform as an operational service; and}
\item{Integration of the components of the Science Platform with each other and with the underlying services on which they run.}

\end{itemize}

\subsection{Features not to be tested}
\label{sec:featnot2test}

This document does not describe facilities for periodically generating or collecting key performance metrics (KPMs), except insofar as those KPMs are incidentally measured as part of executing the documented testcases. 
The KPMs and the system being used to track KPMs and to ensure compliance with documented requirements are described in \citeds{LDM-502}.

\subsection{Pass/fail criteria}
\label{sec:passfail}

The results of all tests will be assessed using the criteria described in \citeds{LDM-503} \S4.

Note that, when executing pipelines, tasks or individual algorithms, any unexplained or unexpected errors or warnings appearing in the associated log or on screen output must be described in the documentation for the system under test. 
Any warning or error for which this is not the case must be filed as a software problem report and filed with the DMCCB.

\subsection{Suspension criteria and resumption requirements}
\label{suspension}

Refer to individual test cases where applicable.

\subsection{Naming convention}

All tests are named according to the pattern \textsc{LSP-xx-yy} where:

\begin{description}

  \item[LSP]{The product under test: the LSST Science Platform}
  \item[xx]{Test specification number (in increments of 10)}
  \item[yy]{Test case number (in increments of 5)}

\end{description}

\section{Test Specifications}

\subsection{LSP-00: Portal and API Aspect Deployment of a Wide-Area Dataset}
\label{lsp-00}

\subsubsection{Objective}

This test demonstrates the deployment of the Portal and API Aspects of the LSST Science Platform to serve all-sky precursor data, including catalogs totaling of order 50-100 billion rows, on the Prototype Data Access Center.
The Prototype Data Access Center relies on a set of hardware and a network environment at NCSA that are part of the Integration Cluster.

The chosen datasets for LSP-00 are from the WISE and NEOWISE missions, 
including all the major catalog data tables from the AllWISE data release, 
as well as the single-epoch source photometry from the original, pre-hibernation mission and from the ``warm'' post-reactivation operations period.
The number of years of post-reactivation data included in the milestone test will be determined by the availability of bulk downloads from IRSA and balancing of the time required to perform the database ingest with other required activities of the WBS 1.02C.06 Data Access group.

LSP-00 also continues the use of the SDSS Stripe 82 (2013 LSST reprocessing) dataset that was used for the earliest Prototype Data Access Center demonstrations [ref: test report].

This test will not include service of the WISE image data in PDAC, as this would require substantial effort to generate a camera model and Butler interface.  
The SDSS data will continue to be used to demonstrate capabilities related to image access.

This test will demonstrate:

\begin{itemize}
\item{Service of a scientifically meaningful all-sky catalog dataset of 50-100 billion rows in Qserv;}
\item{Access to queries against that dataset via a prototype version of the DAX dbserv Web API;}
\item{Interactive access to the data via a prototype of the the Portal Aspect user interface;}
\item{Visualization and exploration of catalog query results in the Portal Aspect;}
\item{Service of image metadata and images via the DAX dbserv and imgserv Web APIs;}
\item{Provision of an image cutout service;}
\item{Visualization of retrieved images and cutouts;}
\item{Linkage of catalog and image data in the Portal Aspect;}
\item{Linkage of related catalog data in the Portal Aspect; and}
\item{Download of query results from the Portal Aspect.}
\end{itemize}

Note that this test specification addresses only functional requirements, and not performance requirements or user-facing performance KPMs.
These will be addressed in subsequent tests.
However, the performance of representative operations will be recorded as part of the test report and will be compared, where feasible, with the performance of comparable queries on the WISE and NEOWISE data in the IRSA archive.

\subsubsection{Test case identification}

\begin{longtable} {|p{0.4\textwidth}|p{0.6\textwidth}|}\hline
\textbf{Test Case}  & \textbf{Description} \\\hline

\hyperref[lsp-00-00]{LSP-00-00} & Verification of the presence of the expected WISE data \\\hline
\hyperref[lsp-00-05]{LSP-00-05} & Demonstration of low-volume and/or indexed queries against the WISE data via API \\\hline
\hyperref[lsp-00-10]{LSP-00-10} & Demonstration of table-scan queries against the WISE data via API \\\hline
\hyperref[lsp-00-15]{LSP-00-15} & Execution of basic catalog queries in the Portal \\\hline
\hyperref[lsp-00-20]{LSP-00-20} & Operation of the UI for interaction with tabular data results \\\hline
\hyperref[lsp-00-25]{LSP-00-25} & Image metadata, image, and image cutout queries \\\hline
\hyperref[lsp-00-30]{LSP-00-30} & Linkage of catalog query results with associated images \\\hline
\hyperref[lsp-00-35]{LSP-00-35} & Linkage of catalog query results to related catalog data \\\hline
% \hyperref[lsp-00-]{LSP-00-} &  \\\hline

\end{longtable}


\section{Test Case Specification}

\subsection{LSP-00-00: Verification of the presence of the expected WISE data}
\label{lsp-00-00}

\subsubsection{Requirements}

The structure of the WISE data allows a test with tables analogous to Object, ForcedSource, and Source, 
so this test partially addresses the implied requirement that Qserv and the API and Portal Aspects be able to handle all the expected data types to be produced by LSST.

Requirement: DMS-LSP-REQ-0001, Access to All Released or Authorized Data Products; DMS-LSP-REQ-0005, Linkage of Aspects (limited to API-Portal linkage)

\subsubsection{Test items}

This test will check:

\begin{itemize}

  \item{That the expected tables are present in the database and accessible via the API Aspect and the Portal Aspect;}
  \item{That the tables are present with the expected schema as documented in the IPAC-provided WISE documentation;}
  \item{That the row counts in the tables are as expected;}
  \item{That the tables cover essentially the entire sky, as expected from the characteristics of the WISE mission.}

\end{itemize}

\subsubsection{Intercase dependencies}

All the further test cases rely on this one being met with near-complete success.

\subsubsection{Environmental needs}

The test must, and can only, be carried out on the PDAC instance of the Science Platform.
It relies on the availability of the IRSA deployment of the WISE and ALLWISE data to use in order to perform comparisons on content and performance.

The test may be performed from any client computer on the public Internet.
The computer used, and the network connection involved, shall be documented in the test report.

The test requires access via the NCSA VPN from the client to the PDAC data services 
(which means that the performance of the VPN is a contributor to the performance observed).

The client computer must have an up-to-date mainstream Web browser, the identity of which shall be documented in the test report.

The API Aspect tests will be captured in a Jupyter notebook run locally on the client computer, so the client system must have a recent production version of Jupyter installed.
(The test does \emph{not} depend on features of JupyterLab in any way, only the standard Notebook.)


\subsubsection{Input specification}

The test verifies, and therefore requires, the presence of the following WISE tables on the PDAC systems:

\begin{itemize}

  \item{The Source Catalog table from the AllWISE data release;}
  \item{The Multi-Epoch Photometry (MEP) table from the AllWISE data release;}
  \item{The Single-exposure Source Database from the WISE All-Sky data release (i.e., from the full-cryo 4-band phase of the original mission);}
  \item{The Single-exposure Source Database from the WISE 3-Band Cryo data release (i.e., from the 3-band, inner-cryogen-tank-only phase of the mission);}
  \item{The Single-exposure Source Database from the NEOWISE Post-Cryo data release (i.e., from the 2-band warm phase of the mission, pre-hibernation);and}
  \item{The Single-exposure Source Database from the NEOWISE Reactivation 2015 data release (i.e., from the first year of the 2-band post-reactivation phase of the mission).}
\end{itemize}


\subsubsection{Output specification}

The output will consist of:

\begin{itemize}
  \item{A Jupyter notebook of the API Aspect tests performed;}
  \item{A written log of the tests performed on the GUI-based Portal Aspect; and}
  \item{A collection of the files that resulted from data download actions performed as part of the tests.}
\end{itemize}


\subsubsection{Procedure}

\begin{enumerate}

  \item{Clone the Github lsst/LDM-540 package.  Record the SHA1 for the version of the package to be used.  [Additional procedures, e.g., tagging, are still to be confirmed.]}
  \item{Log host and networking details of the client host to be used.
 Log the Web browser version to be used.
 Log the version of Jupyter to be used.}
  \item{Establish VPN connectivity to the PDAC at NCSA.}
  \item{Execute the LDM-540/test\textunderscore scripts/lsp-00-00.ipynb notebook to perform the tests listed above against the API Aspect.}
  \item{Preserve any outputs of the script that are in the form of files outside the notebook.}
  \item{Manually perform, and log, the following steps against the Portal Aspect:
    \begin{enumerate}
      \item{Navigate to the PDAC Portal.  Log the URL used to do so.}
      \item{Navigate to the list of WISE catalogs available.  Preserve screen shots as needed to document that the UI offers access to the required catalogs.}
      \item{Perform a trivial query against each required catalog to confirm that the UI does indeed provide access.
 These queries should match ones in the API Aspect test notebook and will generally be small cone searches around (ra=0, dec=0).
 Document the query results with screen shots.
 It is not necessary at this time to exhaustively verify that the Portal provides access to every column of every table; spot checks should be performed and documented.
 (A later test will perform a full column-by-column verification, once test data more closely matching the final DPDD-driven LSST data products are available.)}
%     ...
    \end{enumerate}
  }

\end{enumerate}

\subsection{LSP-00-05: Demonstration of low-volume and/or indexed queries against the WISE data via API}
\label{lsp-00-05}

\subsubsection{Requirements}

???

\subsubsection{Test items}

This test will check that the following low-volume queries can be performed against the WISE catalogs via the API Aspect.

\begin{itemize}

  \item{Small cone searches against the Object-like, ForcedSource-like, and Source-like tables; and}
  \item{Searches by exact ID matching against the Object-like, ForcedSource-like, and Source-like tables.}

\end{itemize}

The tests will record their performance for comparison against similar queries in the production WISE archive at IRSA,
and the returned data will be compared to that for similar queries against the API services provided by IRSA.

\subsubsection{Intercase dependencies}

Several of the queries performed under LSP-00-15, from the Portal Aspect, cover the same ground and will depend on their successful completion under this test case.

\subsubsection{Environmental needs}

The test must, and can only, be carried out on the PDAC instance of the Science Platform.
It relies on the availability of the IRSA deployment of the WISE and ALLWISE data to use in order to perform comparisons on content and performance.

The test may be performed from any client computer on the public Internet.
The computer used, and the network connection involved, shall be documented in the test report.

The test requires access via the NCSA VPN from the client to the PDAC data services 
(which means that the performance of the VPN is a contributor to the performance observed).

The client computer must have an up-to-date mainstream Web browser, the identity of which shall be documented in the test report.

The tests will be captured in a Jupyter notebook run locally on the client computer, so the client system must have a recent production version of Jupyter installed.
(The test does \emph{not} depend on features of JupyterLab in any way, only the standard Notebook.)


\subsubsection{Input specification}

The test requires, the presence of the following WISE tables on the PDAC systems:

\begin{itemize}

  \item{The Source Catalog table from the AllWISE data release;}
  \item{The Multi-Epoch Photometry (MEP) table from the AllWISE data release;}
  \item{The Single-exposure Source Database from the WISE All-Sky data release (i.e., from the full-cryo 4-band phase of the original mission); and}
  \item{The Single-exposure Source Database from the NEOWISE Reactivation 2015 data release (i.e., from the first year of the 2-band post-reactivation phase of the mission).}
\end{itemize}

(The 3-band and 2-band Single-exposure Source Database tables from the pre-hibernation mission are not queried in this test case, for the sake of simplicity.
These tables are smaller than the two Single-exposure Source Database tables that are included in the test case.)

\subsubsection{Output specification}

The output will consist of:

\begin{itemize}
  \item{A Jupyter notebook of the API Aspect tests performed; and}
  \item{A collection of the files that resulted from data download actions performed as part of the tests.}
\end{itemize}


\subsubsection{Procedure}

\begin{enumerate}

  \item{Clone the Github lsst/LDM-540 package.  Record the SHA1 for the version of the package to be used.  [Additional procedures, e.g., tagging, are still to be confirmed.]}
  \item{Log host and networking details of the client host to be used.
 Log the Web browser version to be used.
 Log the version of Jupyter to be used.}
  \item{Establish VPN connectivity to the PDAC at NCSA.}
  \item{Execute the LDM-540/test\textunderscore scripts/lsp-00-05.ipynb notebook to perform the tests listed above against the API Aspect.}
  \item{Preserve any outputs of the script that are in the form of files outside the notebook.}

\end{enumerate}

\subsection{LSP-00-10: Demonstration of table-scan queries against the WISE data via API}
\label{lsp-00-10}

\subsubsection{Requirements}

This test case explores performance of the system against several categories of queries, but does not at this point end with a requirement that the results meet the associated requirements DMS-LSP-REQ-0028 and DMS-LSP-REQ-0029.

\subsubsection{Test items}

This test exercises a range of table-scan-type queries against the WISE data.
Queries shall be performed against the Object-like table, the Forced-Source-like table, and against at least one of the Source-like tables.
A range of query result sizes should be exercised, and shall include at least:

\begin{itemize}

  \item{Queries returning a very small amount of data, fewer than 100 rows, and a small subset of columns;}
  \item{Queries matching a scaled version of the ``low volume'' query definition from the Data Access White Paper; and}
  \item{Queries matching a scaled version of the ``high volume'' query definition from the Data Access White Paper.}

\end{itemize}

The scaling of the ``low volume'' query definition (``50 simultaneous queries against 10 million objects in the catalog, response 10 sec, result data set: ~0.1 GB'') is based on a assumption that the ``against 10 million objects'' is applied against the O(20 billion) rows anticipated in the Object table,
and that it contemplates reducing the scope of any non-indexed portion of the WHERE clause of the query to that fraction of one in $\sim 2000$ of the rows in the table.
Scaled to the $\sim 750$ million rows in the WISE Object-like (AllWISE ``Source Catalog'') table, this would be $\sim 375,000$ rows.
Similarly scaling the result set size suggests a result set of $\sim 3.7$ MB.

Successful completion will be evaluated based on the system's ability to perform the query at all and to return a result with characteristics corresponding to plausible estimates or extrapolations from scaled-down queries against the IRSA WISE archive.
Exact verification may not be realistic because of the lack of a system capable of performing the equivalent queries in the production WISE archive.

At a later date it may be possible to attempt equivalent queries using a non-database system and verify the exact correspondence of results, but the non-database system does not presently exist\footnote{An example of such a system might involve the conversion of one or more WISE all-sky tables to a columnar file-oriented format such as Parquet, and the implementation of queries in Apache Spark or an equivalent system.}.

\subsubsection{Intercase dependencies}

This case depends on the verification by LSP-00-00 of the accessibility of the tables at all.

\subsubsection{Environmental needs}

The test must, and can only, be carried out on the PDAC instance of the Science Platform.
It relies on the availability of the IRSA deployment of the WISE and ALLWISE data to use in order to perform comparisons on content and performance.

The test may be performed from any client computer on the public Internet.
The computer used, and the network connection involved, shall be documented in the test report.

The test requires access via the NCSA VPN from the client to the PDAC data services 
(which means that the performance of the VPN is a contributor to the performance observed).

The client computer must have an up-to-date mainstream Web browser, the identity of which shall be documented in the test report.

The tests will be captured in a Jupyter notebook run locally on the client computer, so the client system must have a recent production version of Jupyter installed.
(The test does \emph{not} depend on features of JupyterLab in any way, only the standard Notebook.)


\subsubsection{Input specification}

This test requires the presence of the following WISE tables on the PDAC systems:

\begin{itemize}

  \item{The Source Catalog table from the AllWISE data release;}
  \item{The Multi-Epoch Photometry (MEP) table from the AllWISE data release;}

\end{itemize}

The results are assumed to apply to the other tables in the PDAC WISE dataset, all of which are comparable or smaller in size.

\subsubsection{Output specification}

The output will consist of:

\begin{itemize}
  \item{A Jupyter notebook of the API Aspect tests performed; and}
  \item{A collection of the files that resulted from data download actions performed as part of the tests.}
\end{itemize}


\subsubsection{Procedure}

\begin{enumerate}

  \item{Clone the Github lsst/LDM-540 package.  Record the SHA1 for the version of the package to be used.  [Additional procedures, e.g., tagging, are still to be confirmed.]}
  \item{Log host and networking details of the client host to be used.
 Log the Web browser version to be used.
 Log the version of Jupyter to be used.}
  \item{Establish VPN connectivity to the PDAC at NCSA.}
  \item{Execute the LDM-540/test\textunderscore scripts/lsp-00-10.ipynb notebook to perform the tests listed above against the API Aspect.}
  \item{Preserve any outputs of the script that are in the form of files outside the notebook.}

\end{enumerate}

\subsection{LSP-00-15: Execution of basic catalog queries in the Portal}
\label{lsp-00-15}

\subsubsection{Requirements}

LSP-level Requirements: DMS-LSP-REQ-0002, [existence of the] Portal Aspect; DMS-LSP-REQ-0014, Download Data

Portal-level Requirements: DMS-PRTL-REQ-0022, Positional Query: Astrophysical Coordinate Systems; DMS-PRTL-REQ-0021, Positional Query: Astrophysical Source Name Lookup; DMS-PRTL-REQ-0021, Positional Query: Multiple Positions/Objects; DMS-PRTL-REQ-0026, Positional Query by Region: Cone-Search; DMS-PRTL-REQ-0027, Positional Query by Region: Box-Search; DMS-PRTL-REQ-0028, Query by Identifier; DMS-PRTL-REQ-0016, Generic Query: Form-based

\subsubsection{Test items}

This test will test the functional requirements to be able to perform a range of basic queries through the Portal Aspect of the LSP:

\begin{itemize}

  \item{Cone searches on the Object-like, ForcedSource-like, and Source-like WISE tables;}
  \item{Multi-target cone searches;}
  \item{Form-based searches for exact equality, e.g., for row IDs; and}
  \item{Form-based searches for sets of object attributes.}

\end{itemize}

In addition, it tests the ability to download tabular query results from the Portal Aspect.

\subsubsection{Intercase dependencies}

As the Portal's implementation of queries of this nature is based on the invocation of the same APIs that will be invoked by LSP-00-05 and LSP-00-10, successful completion of this test case is likely contingent on the success of those two - but does not explicitly depend on them.

Some of the execution of these test cases relies on features of the UI which are formally tested in case LSP-00-20.

\subsubsection{Environmental needs}

The test must, and can only, be carried out on the PDAC instance of the Science Platform.
It relies on the availability of the IRSA deployment of the WISE and ALLWISE data to use in order to perform comparisons on content and performance.

The test may be performed from any client computer on the public Internet.
The computer used, and the network connection involved, shall be documented in the test report.

The test requires access via the NCSA VPN from the client to the PDAC data services 
(which means that the performance of the VPN is a contributor to the performance observed).

The client computer must have an up-to-date mainstream Web browser, the identity of which shall be documented in the test report.

\subsubsection{Input specification}

The test requires the presence of the following WISE tables on the PDAC systems:

\begin{itemize}

  \item{The Source Catalog table from the AllWISE data release;}
  \item{The Multi-Epoch Photometry (MEP) table from the AllWISE data release;}
  \item{The Single-exposure Source Database from the WISE All-Sky data release (i.e., from the full-cryo 4-band phase of the original mission);}
  \item{The Single-exposure Source Database from the NEOWISE Reactivation 2015 data release (i.e., from the first year of the 2-band post-reactivation phase of the mission).}
\end{itemize}


\subsubsection{Output specification}

The output will consist of:

\begin{itemize}
  \item{A written log of the tests performed on the GUI-based Portal Aspect; and}
  \item{A collection of the files that resulted from data download actions performed as part of the tests.}
\end{itemize}


\subsubsection{Procedure}

\begin{enumerate}

  \item{Log host and networking details of the client host to be used.
 Log the Web browser version to be used.}
  \item{Establish VPN connectivity to the PDAC at NCSA.}
  \item{Manually perform, and log, the following steps against the Portal Aspect:
    \begin{enumerate}
      \item{Navigate to the PDAC Portal.  Log the URL used to do so.}
      \item{Perform a cone search around (ra=0,dec=0), radius 300 arcseconds, in each of the Object-like, ForcedSource-like, and a Source-like catalog.
 Choose a row from each search and record the primary key value for each for later use.
 Take screen shots of the search form and of the results of the three searches.
 Record the wall clock time required for the searches, if long enough to measure.}
      \item{Perform a multi-object cone search based on the coordinates in the file LDM-540/test\textunderscore scripts/lsp-00-15.coords in the Object-like table.
 Take screen shots of the search form and of the results of the search.
 Record the wall clock time required for the searches, if long enough to measure.}
      \item{Perform a search on each of the Object-like, ForcedSource-like, and Source-like catalogs for the IDs previously saved.
 Confirm that each search is successful and returns the same information as in the original search from which the ID was taken.
 Perform a search on the ForcedSource-like catalog using the ID from the Object-like catalog.
 Confirm that a time series of measurements of that object in multiple epochs is returned.
 Take screen shots of the search forms and of the results of the searches.
 Record the wall clock time required for the searches, if long enough to measure.}
      \item{On each of the Object-like catalog and a Source-like catalog, by performing searches over small regions of sky and exploring the results, choose a set of attributes and search parameters which should select a relatively small number of rows ($<100,000$) when applied to the entire sky.
 This may require some iterative experimentation at increasingly larger scales.
 Take screen shots of the final search forms and of the results of the searches.
 Record the wall clock time required for the searches, if long enough to measure.}
    \end{enumerate}
  }

\end{enumerate}

\subsection{LSP-00-20: Operation of the UI for interaction with tabular data results}
\label{lsp-00-20}

\subsubsection{Requirements}

???

\subsubsection{Test items}

This test will test the functional requirements to be able to perform certain basic exploratory data analysis functions on tabular data results in the Portal Aspect UI:

\begin{itemize}

  \item{Sort tabular results;}
  \item{Filter tabular results based on the contents of columns;}
  \item{Perform per-row selections from a table;}
  \item{Display 1D histograms of selected attributes;}
  \item{Display 2D scatter plots of selected attributes;}
  \item{Perform graphical selections of rows from plots; and}
  \item{Download tabular query results reflecting sorting and selection.}

\end{itemize}

This test does not address the limits of scaling of these capabilities to large query results.
That will be addressed in future test specifications.
The test report should include notes on the sizes of results that were used.

\subsubsection{Intercase dependencies}

The ability to do these tests depends on the ability to perform tabular queries at all.

\subsubsection{Environmental needs}

The test must, and can only, be carried out on the PDAC instance of the Science Platform.
It relies on the availability of the IRSA deployment of the WISE and ALLWISE data to use in order to perform comparisons on content and performance.

The test may be performed from any client computer on the public Internet.
The computer used, and the network connection involved, shall be documented in the test report.

The test requires access via the NCSA VPN from the client to the PDAC data services 
(which means that the performance of the VPN is a contributor to the performance observed).

The client computer must have an up-to-date mainstream Web browser, the identity of which shall be documented in the test report.


\subsubsection{Input specification}

The test requires the presence of at least the Source Catalog table from the AllWISE data release.


\subsubsection{Output specification}

The output will consist of:

\begin{itemize}
  \item{A written log of the tests performed on the GUI-based Portal Aspect; and}
  \item{A collection of the files that resulted from data download actions performed as part of the tests.}
\end{itemize}


\subsubsection{Procedure}

\begin{enumerate}

  \item{Log host and networking details of the client host to be used.
 Log the Web browser version to be used.}
  \item{Establish VPN connectivity to the PDAC at NCSA.}
  \item{Manually perform, and log, the following steps against the Portal Aspect, recording the time required for a step if it is not trivial:
    \begin{enumerate}
      \item{Navigate to the PDAC Portal.  Log the URL used to do so.}
      \item{Perform a cone search around (ra=0,dec=0), of a radius that produces a result of at least $10,000$ records.
 (Note again that this is a \emph{functionality} test, not a scaling test, however.)}
      \item{Download the query result. 
 Confirm that the download includes the same number of rows reported by the UI.
 Confirm by sampling at the beginning and end that the sort order of the download and that shown in the UI are the same.}
      \item{Exercise sorting of the table, including both numeric and text columns.
 Download the results of performing a sort of each type.
 Confirm by sampling that the sort order of the download matches that in the UI.}
      \item{Verify using external tools (e.g., Unix command-line utilities) that the sort orders for text and numeric columns are appropriate.}
      \item{Exercise filtering of a tabular result based on values of attributes.
 Confirm that it is possible to download the filtered result.
 Verify using external tools that the filter was applied as expected, by comparing the filtered download with the original one.}
      \item{Perform selections of individual rows in the tabular result.
 Confirm that it is possible to download only the selected rows.}
      \item{Construct a 1D histogram of a numeric attribute.
 Using the functionality of the UI, sample a few bin values.
 Verify using external tools that the chosen bin values match those derived from the downloaded data.}
      \item{Construct a 2D scatter plot of selected numeric attributes.
 By inspecting extrema of those attributes in the table, and outlying points in the scatterplot, confirm that the plot appears to contain the expected data.
 (This is not meant to be an exhaustive test of completeness.)}
      \item{Perform a (rectangular) graphical selection from a 2D plot.
 Record the approximate dimensions of the selection as judged from the display.
 Record the exact values for the region boundaries reported by the UI.
 Download the selected data.
 Verify using external tools that the expected selection was applied.}
    \end{enumerate}
  }

\end{enumerate}

\subsection{LSP-00-25: Image metadata, image, and image cutout queries}
\label{lsp-00-25}

\subsubsection{Requirements}

???

\subsubsection{Test items}

This test will check basic functionality related to image search and retrieval, via both the API Aspect and the Portal Aspect of the LSST Science Platform:

\begin{itemize}

  \item{Searching for images containing a specified point;}
  \item{Displaying selected images;}
  \item{Obtaining and displaying image cutouts at a specified point; and}
  \item{Downloading selected images and image cutouts.}

\end{itemize}

Because of limited staff resources, these tests will be based on the original PDAC dataset, the LSST Summer 2013 processing of the SDSS Stripe 82 data.
The image data for the WISE and NEOWISE missions have not been loaded into PDAC.

\subsubsection{Intercase dependencies}

LSP-00-30 relies on functionality explored by this test case.

\subsubsection{Environmental needs}

The test must, and can only, be carried out on the PDAC instance of the Science Platform.

The test may be performed from any client computer on the public Internet.
The computer used, and the network connection involved, shall be documented in the test report.

The test requires access via the NCSA VPN from the client to the PDAC data services 
(which means that the performance of the VPN is a contributor to the performance observed).

The client computer must have an up-to-date mainstream Web browser, the identity of which shall be documented in the test report.

The API Aspect tests will be captured in a Jupyter notebook run locally on the client computer, so the client system must have a recent production version of Jupyter installed.
(The test does \emph{not} depend on features of JupyterLab in any way, only the standard Notebook.)


\subsubsection{Input specification}

The test requires the presence of the following SDSS data, from the Summer 2013 LSST processing, in the PDAC systems:

\begin{itemize}

  \item{The catalog of coadded image tiles;}
  \item{The catalog of calibrated single-epoch images, including photometric zero points;}
  \item{The coadded image data; and}
  \item{The single-epoch calibrated image data.}
\end{itemize}

Note that a significant fraction of the single-epoch calibrated images are known to have been lost between 2013 and the loading of the data into PDAC.
The test report will note where that affects the result.

\subsubsection{Output specification}

The output will consist of:

\begin{itemize}
  \item{A Jupyter notebook of the API Aspect tests performed;}
  \item{A written log of the tests performed on the GUI-based Portal Aspect; and}
  \item{A collection of the files that resulted from data download actions performed as part of the tests.}
\end{itemize}


\subsubsection{Procedure}

\begin{enumerate}

  \item{Clone the Github lsst/LDM-540 package.  Record the SHA1 for the version of the package to be used.  [Additional procedures, e.g., tagging, are still to be confirmed.]}
  \item{Log host and networking details of the client host to be used.
 Log the Web browser version to be used.
 Log the version of Jupyter to be used.}
  \item{Establish VPN connectivity to the PDAC at NCSA.}
  \item{Execute the LDM-540/test\textunderscore scripts/lsp-00-25.ipynb notebook to perform the tests listed above against the API Aspect.}
  \item{Preserve any outputs of the script that are in the form of files outside the notebook.}
  \item{Manually perform, and log, the following steps against the Portal Aspect:
    \begin{enumerate}
      \item{Navigate to the PDAC Portal.  Log the URL used to do so.}
      \item{Perform image queries against the SDSS Stripe 82 coadded and single-epoch image data.
 These queries should match ones in the API Aspect test notebook.
 Document the query results with screen shots.}
      \item{Download an example of each type of image.}
      \item{Using the public SDSS archive, attempt to retrieve corresponding images and visually compare them.}
      \item{Perform image cutout requests for $300\times 300$ fields at the same targets used in the full-image queries.}
      \item{Download the cutouts.}
    \end{enumerate}
  }

\end{enumerate}

\subsection{LSP-00-30: Linkage of catalog query results with associated images}
\label{lsp-00-30}

\subsubsection{Requirements}

???

\subsubsection{Test items}

This test will check for the ability, in the Portal Aspect of the LSST Science Platform, to match catalog data with the image data on which the measurements were performed, specifically:

\begin{itemize}

  \item{Navigating from a catalog query result to the associated images; and}
  \item{Overlaying catalog query results on associated images.}

\end{itemize}

Because of limited staff resources, these tests will be based on the original PDAC dataset, the LSST Summer 2013 processing of the SDSS Stripe 82 data.
The image data for the WISE and NEOWISE missions have not been loaded into PDAC.

\subsubsection{Intercase dependencies}

This test case relies on multiple behaviors explored by other LSP-00-* test cases.

\subsubsection{Environmental needs}

The test must, and can only, be carried out on the PDAC instance of the Science Platform.

The test may be performed from any client computer on the public Internet.
The computer used, and the network connection involved, shall be documented in the test report.

The test requires access via the NCSA VPN from the client to the PDAC data services 
(which means that the performance of the VPN is a contributor to the performance observed).

The client computer must have an up-to-date mainstream Web browser, the identity of which shall be documented in the test report.


\subsubsection{Input specification}

The test requires the presence of the following SDSS data, from the Summer 2013 LSST processing, in the PDAC systems:

\begin{itemize}

  \item{The coadded source catalog;}
  \item{The forced photometry catalog driven from the $i$-band coadded sources;}
  \item{The catalog of coadded image tiles;}
  \item{The catalog of calibrated single-epoch images, including photometric zero points;}
  \item{The coadded image data; and}
  \item{The single-epoch calibrated image data.}
\end{itemize}

Note that a significant fraction of the single-epoch calibrated images are known to have been lost between 2013 and the loading of the data into PDAC.
The test report will note where that affects the result.

\subsubsection{Output specification}

The output will consist of:

\begin{itemize}
  \item{A written log of the tests performed on the GUI-based Portal Aspect.}
\end{itemize}


\subsubsection{Procedure}

\begin{enumerate}

  \item{Log host and networking details of the client host to be used.
 Log the Web browser version to be used.}
  \item{Establish VPN connectivity to the PDAC at NCSA.}
  \item{Manually perform, and log, the following steps against the Portal Aspect:
    \begin{enumerate}
      \item{Navigate to the PDAC Portal.  Log the URL used to do so.}
      \item{Perform a catalog query against the i-band coadded source catalog within the Stripe 82 area.}
      \item{Request the time series from the forced photometry catalog for a selected source.}
      \item{Confirm that the images associated with the forced photometry measurements are available.}
      \item{Document the query results with screen shots.}
    \end{enumerate}
  }

\end{enumerate}

\subsection{LSP-00-35: Linkage of catalog query results to related catalog data}
\label{lsp-00-35}

\subsubsection{Requirements}

Requirement: DMS-LSP-REQ-0008, Semantic Linkage

\subsubsection{Test items}

This test will check for the ability, in the Portal Aspect of the LSST Science Platform, to match catalog data with related catalog data.  
Specifically, the test verifies the ability to navigate from a coadded source catalog entry to the associated forced photometry.


\subsubsection{Intercase dependencies}

This test case relies on multiple behaviors explored by other LSP-00-* test cases.


\subsubsection{Environmental needs}

The test must, and can only, be carried out on the PDAC instance of the Science Platform.

The test may be performed from any client computer on the public Internet.
The computer used, and the network connection involved, shall be documented in the test report.

The test requires access via the NCSA VPN from the client to the PDAC data services 
(which means that the performance of the VPN is a contributor to the performance observed).

The client computer must have an up-to-date mainstream Web browser, the identity of which shall be documented in the test report.


\subsubsection{Input specification}

The test requires the presence of the following SDSS data, from the Summer 2013 LSST processing, in the PDAC systems:

\begin{itemize}

  \item{The coadded source catalog; and}
  \item{The forced photometry catalog driven from the $i$-band coadded sources.}

\end{itemize}


\subsubsection{Output specification}

The output will consist of:

\begin{itemize}
  \item{A written log of the tests performed on the GUI-based Portal Aspect.}
\end{itemize}


\subsubsection{Procedure}

\begin{enumerate}

  \item{Log host and networking details of the client host to be used.
 Log the Web browser version to be used.}
  \item{Establish VPN connectivity to the PDAC at NCSA.}
  \item{Manually perform, and log, the following steps against the Portal Aspect:
    \begin{enumerate}
      \item{Navigate to the PDAC Portal.  Log the URL used to do so.}
      \item{Perform a catalog query against the i-band coadded source catalog within the Stripe 82 area.}
      \item{Request the time series from the forced photometry catalog for a selected source.}
      \item{Confirm that the data retrieved corresponds to the selected coadded source.}
      \item{Document the query results with screen shots.}
    \end{enumerate}
  }

\end{enumerate}



\appendix

\section{Precursor test datasets}

\subsection{Summer 2013 SDSS Stripe 82 Processing}

\subsection{Data from the WISE and NEOWISE missions}

\end{document}
